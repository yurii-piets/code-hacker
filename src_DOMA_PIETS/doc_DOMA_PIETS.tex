\documentclass[a4paper,12pt]{article}
%%%%% Pakiety
%% Kodowanie dokumentu
\usepackage{polski}
\usepackage[utf8]{inputenc}
\usepackage[T1]{fontenc}
%% Opcje wyświetlania
\usepackage{times}
\usepackage{anysize}
\usepackage{indentfirst}
\usepackage{hyperref}
%% Reszta narzędzi
\usepackage{xcolor}
\usepackage{graphicx}
\usepackage{listings}

%%%%% Opcje dokumentu
\title{Projekt z mikrokontrolerów - sprawozdanie}
	
\marginsize{2cm}{2cm}{2cm}{2cm}
\sloppy

%%%%% Opcje hiperłącza
\hypersetup{
	colorlinks,
	linkcolor={red!50!black},
	citecolor={blue!50!black},
	urlcolor={blue!80!black}
}

%%%%% Do listingu kodu
\lstdefinestyle{customc}{
%%% Kształt całości
	belowcaptionskip=1\baselineskip,
	breaklines=true,
	frame=L,
	xleftmargin=\parindent,
	language=C,
	showstringspaces=false,
%%% Numery linii kodu
	numberfirstline=true,
	numbers=left,
	stepnumber=5,    
	firstnumber=1,
%%% Formatowanie kolorów i czcionki
	basicstyle=\footnotesize\ttfamily,
	morekeywords = {boolean, byte, inline},
	keywordstyle=\bfseries\color{green!40!black},
	commentstyle=\itshape\color{purple!40!black},
	identifierstyle=\color{blue},
	stringstyle=\color{orange},
%%% Polskie znaki
	inputencoding=utf8x, 
	extendedchars=\true,
	literate={ą}{{\k{a}}}1
				{Ą}{{\k{A}}}1
				{ę}{{\k{e}}}1
				{Ę}{{\k{E}}}1
				{ó}{{\'o}}1
				{Ó}{{\'O}}1
				{ś}{{\'s}}1
				{Ś}{{\'S}}1
				{ł}{{\l{}}}1
				{Ł}{{\L{}}}1
				{ż}{{\.z}}1
				{Ż}{{\.Z}}1
				{ź}{{\'z}}1
				{Ź}{{\'Z}}1
				{ć}{{\'c}}1
				{Ć}{{\'C}}1
				{ń}{{\'n}}1
				{Ń}{{\'N}}1
}

\lstset{style=customc}

%%%%%%%%%%%%%%%
\begin{document}

\maketitle

%%%%%%%%%%%%%%%
\section{Informacje ogólne}
\par 1. Temat: Łamanie kodu 
\par 2. Numer grupy: 1b
\par 3. Sklad osobowy: Sebastian Domarecki, Yurii Piets
\par 4. Kierunek: informatyka
\par 5. Rok studiów: II
\par 6. Rok akademicki: 2016-2017

%%%%%%%%%%%%%%%
\section{Opis działania}
%%%
\subsection{Schemat połączenia}

\begin{figure}[!ht]
	
	\centering
	\includegraphics[scale=0.5]{Mikro_bb.png}
	\qquad
	\begin{tabular}[b]{cc}\hline
		TM1638 & Arduino \\ \hline
		VCC & 5V \\
		GND & GND \\
		STB & Digital 7 \\
		CLK & Digital 9 \\
		DIO & Digital 8 \\ \hline
	\end{tabular}
	\label{fig:schemat}
\end{figure}
%%%
\subsection{Opis algorytmu}
\par Schemat działania algorytmu opiera się w głównej mierze na zredukowanej zewnętrznej bibliotece TM1638.h udostępnianej na \href{https://github.com/rjbatista/tm1638-library}{Githubie}.
\par Program po inicjalizacji ekranu stringiem ośmiu spacji, aż do zakończenia łamania kodu znajduje się w pętli iterowanej po numerze dekodowanej cyfry.\\
Wewnątrz niej kolejna pętla for TIMES razy zmienia wyświetlane wartości niezdekodowanych cyfr na wygenerowane przypadowe dozwolone.\\
Za każdą iteracją drugiej pętli sprawdzane jest w sposób ciągły przez DISP czasu czy został wciśnięty przycisk, bądź wprowadzona komenda przez port szeregowy.\\
Po jej opuszczeniu włączany jest kolejny led oraz kopiowana wartość zdekodowanej cyfry z CODE do tablicy display do wyświetlenia.
\par HandleClick w zależności od stanu programu pozwala na jego zmianę po wykryciu sygnału z odpowiedniego przycisku.
Następnie program czeka na zwolnienie wszystkich przycisków.
\par ReadInput najpierw czeka na zapełnienie buforu, następnie wykrywa rodzaj komendy i przekierowuje ruch do odpowiedniej funkcji. Dla błędnego polecenia wypisywany jest stosowny komunikat.
%%%
\subsection{Elementy programu}

\subsubsection{Zmienne}
\par CODE,TIMES,DISP - zgodnie z wytycznymi.
\par module - obiekt klasy TM1638 udostępniający prosty interfejs do obsługi płytki
\par state - zmienna typu wyliczeniowego wyświetlająca stan programu - dostępne IN\_PROGRESS, WAITING, FINISHED, RESET
\subsubsection{Stałe}
\par strobe,clock,data - piny Arduino łączące się z płytką TM1638
\par DISPLAY\_SIZE - wielkość wyświetlacza płytki
\par allowed\_chars - tablica zawierająca dozwolone do wyświetlenia znaki, wszystkie inne są odrzucone
\subsubsection{Funkcje}
\par void handleClick(states *) - obsługa przycisków
\par void readInput(states *) - obsługa poleceń z łącza szeregowego
\par boolean initCode() , initTimes(), initDisp() - wywoływane przy readInput dla pierwszego przesłanego znaku odpowiednio C, N lub D, sczytują z łącza szeregowego nową wartość parametru pracy programu
\par boolean isAllowed(char) - pomocnicza funkcja sprawdzająca czy argument jest dozwolonym znakiem
\par inline void waitTillRelease() - pomocnicza funkcja czekająca na zwolnienie wszystkich przycisków

%%%
\subsection{Używane biblioteki}
Autorsko zredukowana wersja TM1638.h udostępniana na licencji GNU GPL v3 przez Ricardo Batista.\\
Link: \href{https://github.com/rjbatista/tm1638-library}{https://github.com/rjbatista/tm1638-library}

%%%%%%%%%%%%%%%
\section{Listing kodu}
\lstinputlisting{src_DOMA_PIETS.ino}

%%%%%%%%%%%%%%%
\end{document}
